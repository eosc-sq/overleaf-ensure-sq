\textbf{Editing}: Miguel Colom

\textbf{Contributors}: Laura del Cano

\subsection{Developer}

\miguel{Are the authors of the research SW obtaining an official acknowledgement of the value of their work? Does this have an impact on
their career?}

\miguel{Which are the licenses available? Which one should the developer use? Are the developers free to pick any SW license, or is it their
employer who picks it?}

\miguel{Does the developers of research SW have all the rights about the research SW they produce? Intellectual property (authoring),
exploitation, re-use, re-licensing.}

\subsection{User}

\miguel{Documentation: how to install, and use the SW. Examples.}

\miguel{How to cite research SW?}

\miguel{Is the research SW easy to understand and modify? Is it possible to do so? SW licenses.}

\miguel{Level of reproducibility of the research SW.}

\subsection{Service provider}

\miguel{Examples of service providers for research SW: Galaxy project, Jupyter lab, IEEE's Code Ocean, supercomputing infrastructures
(BSC, and others), CERN data analysis platform, ...}

\miguel{Can the SW be put inside a container?}

\miguel{Are the developers providing timely updates (bug fixes, security updates)}

Abdulrahman Azab: See how/where these topics fit
Portability (containerization) - probably should be included in 04-SW in production
Define an agreed upon procedure for SW packaging. What to have as containers/conda packages, what to have as e.g. Easybuild recipes. Should we promote one portable solution or a combination of several?
Reproducibility, portability, archiving



\cerlane{Question: Are we talking about service providers of the Infrastructure or research SW?}
