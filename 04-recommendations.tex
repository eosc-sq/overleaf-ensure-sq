\subsection{Quality Attributes recommendations}

% \violaine{It remains to make the link with the quality criteria defined in part 1}

\subsubsection{Individual researcher creating software for personal use} 

The expected quality criteria are minimal but essential to ensure
reuse and continuity of developments (usually by another doctoral
student and/or by the research team), as well as the possibility of
reproducing the results of associated publications. 

{\bf Practices:}
\begin{itemize}
\item Version Control System
\item Basic documentation Documentation de base (including comments in the code)
\item License 
\item archiving and citation
\item Publication on the code
\end{itemize}

{\bf Tools and infrastructure required:}
\begin{itemize}
\item Software forge (with public access) - For example gitlab instance
\item Automatic documentation generator - For example doxygen
\item Archiving infrastructure (with perennial id) - For example Software Heritage
\item Cataloguing infrastructure - For example HAL en France
\item Journal for the publication on codes - For example JOSS
\end{itemize}

% \violaine{Add needed skills -> need training at PhD, researcher and engineer level \\
% Add needed support -> help to identify the adapted infrastructures and how to use them}

\subsubsection{A research team creating an application or workflow for use within the team}

The quality criteria are more demanding because it is a question here
of ensuring continuity in the development carried out by several
people, over more or less long periods (researchers, engineers,
doctoral students, trainees...).

In addition to the above elements, more advanced practices must be
integrated.

{\bf Additional practices:}
\begin{itemize}
\item Release management
\item Testing (unit, functional, regression)
\item Continuous integration
\item Programming rules
\item Bug tracking
\item License management
\end{itemize}

{\bf Additional tools and infrastructure required:}
\begin{itemize}
\item Continuous integration module - For example Jenkins or gitlab CI
\item Issue tracking system - For example Jira, Redmine ...
\item Static web site - For example gitlab pages 
\item License compatibility tool - For example flict
\end{itemize}

% \violaine{Add needed skills and support as about license, CI ...}

\subsubsection{A team / community developing open source software}

The software is developed within the framework of a community,
generally distributed, whose actors have different roles. In general,
there is a software governance that organizes the developments. The
aim is to ensure optimal quality so that the whole community can use
the software with confidence and also make contributions.

{\bf Additional practices:}
\begin{itemize}
\item Easy installation
\item User and developer documentation
\item Contribution tracking / Copyright management
\item Regular releases
\item Code review
\item Code coveraging
\item Code quality check
\item Possibility of online tutorials / avaibility of examples
\item User support
\item Ensure visibility
\item Ensure good development practices and compliance with standards
\end{itemize}

{\bf Additional tools and infrastructure required:}
\begin{itemize}
\item Software forge allowing contributions to the largest number
\item Test server - For example Cdash ...
\item Packaging system (such as guix, nix, spack ...) / VM with these systems installed
\item Container / Kubernetes
\item Docker registry
\item Code coverage tool
\item Static code analysis tool
\item Notebook server
\item Forum / chat tool
\item Code quality verification tool - For example SonarQube
\item DCO (Developer Certificate of Origin) or CLA (Contributor License Agreement) / Copyright
\item Synchronization between forges - For example Fediverse, Forge Friends ...
\item Tool for detecting bad development practices / non compliance with
standards
\end{itemize}

Software examples - by far not exhaustive - in different communities
are: Astropy1 (astronomy);  ESMValTool2 (Earth sciences); VASP3,
FHI-aims4 (material research); VIP5 (medical sciences); Gromacs6
(molecular dynamics); Neuron7, Nest8, Genesis (Neuroinformatics); root
(particle physics); many more examples can be found at, e.g., the
Research Software Directory

% \violaine{Link with the computing centers for support of packaging system or container}

\subsubsection{A team or community creating a service, a platform or an infrastructure}

The case is quite similar to the previous one but includes additional
constraints linked to the provision of a service. The necessary means
in operational terms are more important. 

{\bf Additional practices:}
\begin{itemize}
\item Viable business model
\item Security of the service/platform
\end{itemize}

{\bf Additional tools and infrastructure required:}
\begin{itemize}
\item Storage
\item VM / Container
\end{itemize}

\subsubsection{A team / community developing software in an industrial context}

The expected quality criteria are very similar to the case of
community developments. Depending on the nature of the collaboration
with the industry, the security aspects may become essential. 

{\bf Additional practices:}
\begin{itemize}
\item Code certification
\item Dependency security
\end{itemize}

{\bf Additional tools and infrastructure required:}
\begin{itemize}
\item Certification Infrastructure
\item Vulnerability Detection Tool
\end{itemize}

\subsection{Example of tools, services and infrastructures to implement Quality Assurance for RS}

% \violaine{The idea is to refer to some search software of the different categories and show the similarities and differences with
% the analysis of the top to point out the possible lacks in the current service offer} 
