\textbf{Editing}: Mario David, Daniel Garijo

\textbf{Contributors}: daniel.garijo@upm.es, miguel.colom-barco@ens-paris-saclay.fr, elisabetta.ronchieri@cnaf.infn.it (add yourself here)

\miguel{Indeed,having a section in the final document where we give definitions of "quality", "software", "research software", etc. is
quite convenient. It doesn't mean other definitions are not possible, but we need to settle clearly what we're discussing in our own context.}

Criteria of quality (brainstorming, to complete by everybody)

\miguel{For scientific research: reviewed by other peers, reproducible (data + sources, matching the article), Open Access.}

\dg{We should look at the FAIR4RS reports for the alignment for quality dimensions. In a sense, this is what FAIR aims (at a basic level).}

\miguel{One quality criteria: reproducibility. Define "reproducibility" and "repeatability". Different levels of reproducibility. Bottom level:
black box (being able to obtain the same results given the same inputs). Higher level: full access to a well-commented source code, and checking
that the pseudocode matches well the implementation. In the case of a scientific publication, checking that the implementation matches exactly
what the descriptions in the article.}

\dg{We should first define what we understand by software. We can reuse the definition in FAIR4RS.}

\mdavid{example of SQA criteria matrix comparison, note that the SQA baseline was an old version v3.1, it’s now at v4.0:}

\url{https://docs.google.com/document/d/1GNq06zfUr\_6Sb0YWnA7q\_DKbS9liICRq3UpVX-DIGDQ/edit}

\mdavid{check section 2.1 State of the Art, elisabetta.ronchieri@cnaf.infn.itto coordinate
\url{https://digital.csic.es/bitstream/10261/219306/1/EOSC-SYNERGY-WP3-D3.1.pdf}}

\dg{Missing the initial table here, although we had only summaries. We should see how each of those papers defined software quality dimensions.}

\EliR{Concerning Quality Definition, there are various software models and standards that can be referred. However, the most recent ones are ISO and IEEE. Existing models and standards provide a full set of software characteristics: unfortunately they quite often have different meaning and do not have a match with software metrics used to measure software characteristics.}




\subsection{Software Quality Dimensions: A survey}

\dg{Text introducing the section}

\subsubsection{Survey Methodology}
\dg{I have done a quick summary, draft. Each of the points and descriptions should be extended}
We follow the methodology proposed by Kitchenham and Charters \cite{kitchenham2007guidelines} which has the following steps:
\begin{enumerate}
    \item Source selection and search: We have searched in the Scopus dataset, including the top five journals in software engineering related to software \footnote{\url{https://research.com/journals-rankings/computer-science/software-programming}} and articles of the  "International Conference on Software Engineering", one of the top venues for Software engineering \dg{citation needed}. \dg{Also, add here something about the books we considered, or internal knowledge by the authors}. The search included the keywords "software quality" in the title of the target publications. \dg{this can be improved, but we already obtain thousands of results}
    \item Inclusion and exclusion criteria: Excluded journals not in the SE domain. Excluded articles not written in English.
    \item Selection procedure: \dg{skim article titles and abstracts. The process was performed by 2-3 people.Final list was agreed upon by the group. Explain what the agreement procedure was}
    \item Review process: \dg{After following the selection procedure, we ended up with XX articles, which have been reviewed in this survey}
\end{enumerate}

Journals: IEEE Transactions on Software Engineering, Empirical Software Engineering, Journal of Systems and Software, Software \& Systems Modeling,  Information and Software Technology, IEEE Software, Software Quality Journal. Query used:

\tiny
\begin{verbatim}
    TITLE ( software  AND quality )  AND  
        ( LIMIT-TO ( EXACTSRCTITLE ,  "Software Quality Journal" )  
        OR  LIMIT-TO ( EXACTSRCTITLE ,  "Proceedings International Conference On Software Engineering" ) 
        OR  LIMIT-TO ( EXACTSRCTITLE ,  "IEEE Transactions on Software Engineering" )
        OR  LIMIT-TO ( EXACTSRCTITLE ,  "Empirical Software Engineering" ) 
        OR  LIMIT-TO ( EXACTSRCTITLE ,  "Journal of Systems and Software" ) 
        OR  LIMIT-TO ( EXACTSRCTITLE ,  "Software & Systems Modeling" ) 
        OR  LIMIT-TO ( EXACTSRCTITLE ,  "Information and Software Technology" )  
        OR  LIMIT-TO ( EXACTSRCTITLE ,  "IEEE Software" )   
        )  AND  ( LIMIT-TO ( SUBJAREA ,  "COMP" )  OR  LIMIT-TO ( SUBJAREA ,  "ENGI" ) )  
\end{verbatim}
\small

As a result, we get 272 results. Additional filtering: excluding papers with no abstracts, proceedings/workshop summary, and removed those which did not seem related by browsing the abstract and title. Also, removed those papers that did not seem to propose quality dimensions (e.g., if they talk about practices)

%Query: \footnote{\url{https://www.scopus.com/results/results.uri?sort=plf-f&src=s&nlo=&nlr=&nls=&sid=b04451f7b887660ce99d73bfdbdc4fc8&sot=a&sdt=a&cluster=scosubjabbr%2c%22COMP%22%2ct%2c%22ENGI%22%2ct%2bscoexactsrctitle%2c%22Software+Quality+Journal%22%2ct%2c%22Proceedings+International+Conference+On+Software+Engineering%22%2ct%2c%22IEEE+Transactions+on+Software+Engineering%22%2ct%2c%22Empirical+Software+Engineering%22%2ct%2c%22Journal+of+Systems+and+Software%22%2ct%2c%22Software+%26+Systems+Modeling%22%2ct%2c%22Information+and+Software+Technology%22%2ct%2c%22IEEE+Software%22%2ct&sl=30&s=TITLE+%28+software+AND+quality+%29&cl=t&offset=201&origin=resultslist&ss=plf-f&ws=r-f&ps=r-f&cs=r-f&cc=10&txGid=cff6749bf8a084912200dbef379950d4} 


%Also, this book: https://books.google.es/books?hl=en&lr=&id=XTvpAQAAQBAJ&oi=fnd&pg=PR3&dq=software+quality&ots=fohz_-KW0d&sig=5TGlvR3sgAIkAHzs5Iup8Qijpuo#v=onepage&q=software%20quality&f=false looks nice!

\subsection{Software Quality Characteristics}

\textcolor{red}{Roberto suggested having a look at: https://www.niso.org/press-releases/2021/01/nisos-recommended-practice-reproducibility-badging-and-definitions-now as well}

\begin{center}
\tablefirsthead{}
\tablehead{}
\tabletail{}
\tablelasttail{}
\caption{Software Quality characteristics are taken from ISO/IEC
25010:2011(E)~\cite{1400-1700_isoiec_nodate} except Scalability and Supportability.
Microsoft: \url{https://docs.microsoft.com/en-us/previous-versions/msp-n-p/ee658094(v=pandp.10)}.}
\label{tab:table1}
\tiny
\begin{supertabular}{|p{0.15\linewidth}|p{0.4\linewidth}|p{0.4\linewidth}|}
\hline
\textbf{Characteristic} & \textbf{Definition} & \textbf{Notes} \\
\hline

Functional suitability &
Degree to which a product or system provides functions that meet stated and implied needs when used under specified conditions. &
-
\\ \hline

Availability &
Degree to which a system, product or component is operational and accessible when required for use. &
Adapted from ISO/IEC/IEEE 24765
\\ \hline

Reliability &
Degree to which a system, product or component performs specific functions under specified conditions for a specified period of time. &
Adapted from ISO/IEC/IEEE 24765. Limitations in reliability are due to faults in requirements, design and implementation, or due to contextual changes.
\\ \hline

Time behaviour &
Degree to which the response and processing times and throughput rates of a product or system, when performing its functions, meet requirements. &
-
\\ \hline

Performance &
Performance relative to the amount of resources used under stated conditions. &
Resources can include other software products, the software and hardware configuration of the system, and materials (e.g. print paper, storage media).
\\ \hline

Ease of use (Usability) &
Degree to which a product or system can be used by specified users to achieve specific goals with effectiveness, efficiency and satisfaction in a specified context of use. &
Adapted from ISO 9241-210. Usability can either be specified or measured as a product quality characteristic in terms of its sub-characteristics, or specified or measured directly by measures that are a subset of quality in use.
\\ \hline

Fault tolerance &
Degree to which a system, product or component operates as intended despite the presence of hardware or software faults. &
Adapted from ISO/IEC/IEEE 24765.
\\ \hline

Security &
Degree to which a product or system protects information and data so that persons or other products or systems have the degree of data access appropriate to their types and levels of authorisation. &
As well as data stored in or by a product or system, security also applies to data in transmission. Survivability (the degree to which a product or system continues to fulfil its mission by providing essential services in a timely manner in spite of the presence of attacks) is covered by recoverability (4.2.5.4). Immunity (the degree to which a product or system is resistant to attack) is covered by integrity (4.2.6.2). Security contributes to trust (4.1.3.2).
\\ \hline

Maintainability &
Degree of effectiveness and efficiency with which a product or system can be modified by the intended maintainers. &
Modifications can include corrections, improvements or adaptation of the software to changes in environment, and in requirements and functional specifications. Modifications include those carried out by specialised support staff, and those carried out by business or operational staff, or end users. Maintainability includes installation of updates and upgrades. Maintainability can be interpreted as either an inherent capability of the product or system to facilitate maintenance activities, or the quality in use experienced by the maintainers for the goal of maintaining the product or system.
\\\hline

Recoverability &
Degree to which, in the event of an interruption or a failure, a product or system can recover the data directly affected and re-establish the desired state of the system. &
Following a failure, a computer system will sometimes be down for a period of time, the length of which is determined by its recoverability.
\\ \hline

Operability / Manageability &
Degree to which a product or system has attributes that make it easy to operate and control. &
Operability corresponds to controllability, (operator) error tolerance and conformity with user expectations as defined in ISO 9241-110.
\\ \hline

Resource utilisation &
Degree to which the amounts and types of resources used by a product or system, when performing its functions, meet requirements. &
Human resources are included as part of efficiency (4.1.2).
\\ \hline

Safety &
Degree to which a product or system mitigates the potential risk to people in the intended contexts of use. &
-
\\ \hline

Interoperability &
Degree to which two or more systems, products or components can exchange information and use the information that has been exchanged. &
Based on ISO/IEC/IEEE 24765.
\\ \hline

Attractiveness &
Renamed as user interface aesthetics. Degree to which a user interface enables pleasing and satisfying interaction for the user. &
This refers to properties of the product or system that increase the pleasure and satisfaction of the user, such as the use of colour and the nature of the graphical design.
\\ \hline

Compatibility &
Degree to which a product, system or component can exchange information with other products, systems or components, and/or perform its required functions, while sharing the same hardware or software environment. &
Adapted from ISO/IEC/IEEE 24765.
\\ \hline

Installability &
Degree of effectiveness and efficiency with which a product or system can be successfully installed and/or uninstalled in a specified environment. &
-
\\ \hline

Technical accessibility &
Degree to which a product or system can be used by people with the widest range of characteristics and capabilities to achieve a specified goal in a specified context of use. &
The range of capabilities includes disabilities associated with age. Accessibility for people with disabilities can be specified or measured either as the extent to which a product or system can be used by users with specified disabilities to achieve specified goals with effectiveness, efficiency, freedom from risk and satisfaction in a specified context of use, or by the presence of product properties that support accessibility.
\\ \hline

Portability /Adaptability &
Degree of effectiveness and efficiency with which a system, product or component can be transferred from one hardware, software or other operational or usage environment to another. &
Adapted from ISO/IEC/IEEE 24765. Portability can be interpreted as either an inherent capability of the product or system to facilitate porting activities, or the quality in use experienced for the goal of porting the product or system.
\\ \hline

Modifiability &
Degree to which a product or system can be effectively and efficiently modified without introducing defects or degrading existing product quality. &
Implementation includes coding, designing, documenting and verifying changes. Modularity (4.2.7.1) and analysability (4.2.7.3) can influence modifiability. Modifiability is a combination of changeability and stability.
\\ \hline

Reusability &
Degree to which an asset can be used in more than one system, or in building other assets. &
-
\\ \hline

Scalability &
1.) Scalability is the ability of a system to either handle increases in load without impact on the performance of the system, or the ability to be readily enlarged $OR$ 2.) Scalability is the capability of algorithms, protocols, and applications to efficiently handle a growing amount of work or the demand of increasing its performance (according to some metrics) by adding resources to the system in which the software is running. Resources can be added to the single nodes (vertical scalability) and to the system as a whole (horizontal scalability). &
1.) Microsoft, 2.) Massimo Torquati
\\ \hline

Supportability &
1.) Supportability is the ability of the system to provide information helpful for identifying and resolving issues when it fails to work correctly $OR$ 2.) Existence of a helpdesk or issue tracking, bug reporting, enhancements and general support.  &
1.) Microsoft, 2.) Mario David
\\ \hline

Testability &
Degree of effectiveness and efficiency with which test criteria can be established for a system, product or component and tests can be performed to determine whether those criteria have been met. &
Adapted from ISO/IEC/IEEE 24765.
\\ \hline

Confidentiality &
Degree to which a product or system ensures that data are accessible only to those authorised to have access. &
-
\\ \hline

\end{supertabular}
\end{center}
