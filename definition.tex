\textbf{Editing}: Mario David

\textbf{Contributors}: daniel.garijo@upm.es, miguel.colom-barco@ens-paris-saclay.fr, elisabetta.ronchieri@cnaf.infn.it (add yourself here)

\miguel{Indeed,having a section in the final document where we give definitions of "quality", "software", "research software", etc. is
quite convenient. It doesn't mean other definitions are not possible, but we need to settle clearly what we're discussing in our own context.}

Criteria of quality (brainstorming, to complete by everybody)

\miguel{For scientific research: reviewed by other peers, reproducible (data + sources, matching the article), Open Access.}

\dg{We should look at the FAIR4RS reports for the alignment for quality dimensions. In a sense, this is what FAIR aims (at a basic level).}

\miguel{One quality criteria: reproducibility. Define "reproducibility" and "repeatability". Different levels of reproducibility. Bottom level:
black box (being able to obtain the same results given the same inputs). Higher level: full access to a well-commented source code, and checking
that the pseudocode matches well the implementation. In the case of a scientific publication, checking that the implementation matches exactly
what the descriptions in the article.}

\dg{We should first define what we understand by software. We can reuse the definition in FAIR4RS.}

\mdavid{example of SQA criteria matrix comparison, note that the SQA baseline was an old version v3.1, it’s now at v4.0:}

\url{https://docs.google.com/document/d/1GNq06zfUr\_6Sb0YWnA7q\_DKbS9liICRq3UpVX-DIGDQ/edit}

\mdavid{check section 2.1 State of the Art, elisabetta.ronchieri@cnaf.infn.itto coordinate
\url{https://digital.csic.es/bitstream/10261/219306/1/EOSC-SYNERGY-WP3-D3.1.pdf}}

\dg{Missing the initial table here, although we had only summaries. We should see how each of those papers defined software quality dimensions.}

\EliR{Concerning Quality Definition, there are various software models and standards that can be referred. However, the most recent ones are ISO and IEEE. Existing models and standards provide a full set of software characteristics: unfortunately they quite often have different meaning and do not have a match with software metrics used to measure software characteristics.}




\subsection{Software Quality Dimensions: A survey}

\dg{Text introducing the section}

\subsubsection{Survey Methodology}
\dg{quick summary, draft. Each of the points and descriptions should be extended}
We follow the methodology proposed by Kitchenham and Charters \cite{kitchenham2007guidelines} which has the following steps:
\begin{enumerate}
    \item Source selection and search: We have searched in the Scopus dataset, including the top five journals in software engineering related to software \footnote{\url{https://research.com/journals-rankings/computer-science/software-programming}} and articles of the  "International Conference on Software Engineering", one of the top venues for Software engineering \dg{citation needed}. \dg{Also, add here something about the books we considered, or internal knowledge by the authors}. The search included the keywords "software quality" in the title of the target publications. \dg{this can be improved, but we already obtain thousands of results}
    \item Inclusion and exclusion criteria: Excluded journals not in the SE domain. Excluded articles not written in English.
    \item Selection procedure: \dg{skim article titles and abstracts. The process was performed by 2-3 people.Final list was agreed upon by the group. Explain what the agreement procedure was}
    \item Review process: \dg{After following the selection procedure, we ended up with XX articles, which have been reviewed in this survey}
\end{enumerate}

Journals: IEEE Transactions on Software Engineering, Empirical Software Engineering, Journal of Systems and Software, Software \& Systems Modeling,  Information and Software Technology, IEEE Software, Software Quality Journal. Query used:

\begin{verbatim}
    TITLE ( software  AND quality )  AND  
        ( LIMIT-TO ( EXACTSRCTITLE ,  "Software Quality Journal" )  
        OR  LIMIT-TO ( EXACTSRCTITLE ,  "Proceedings International Conference On Software Engineering" ) 
        OR  LIMIT-TO ( EXACTSRCTITLE ,  "IEEE Transactions on Software Engineering" )
        OR  LIMIT-TO ( EXACTSRCTITLE ,  "Empirical Software Engineering" ) 
        OR  LIMIT-TO ( EXACTSRCTITLE ,  "Journal of Systems and Software" ) 
        OR  LIMIT-TO ( EXACTSRCTITLE ,  "Software & Systems Modeling" ) 
        OR  LIMIT-TO ( EXACTSRCTITLE ,  "Information and Software Technology" )  
        OR  LIMIT-TO ( EXACTSRCTITLE ,  "IEEE Software" )   
        )  AND  ( LIMIT-TO ( SUBJAREA ,  "COMP" )  OR  LIMIT-TO ( SUBJAREA ,  "ENGI" ) )  
\end{verbatim}

As a result, we get 272 results. Additional filtering: excluding papers with no abstracts, proceedings/workshop summary, and removed those which did not seem related by browsing the abstract and title. Also, removed those papers that did not seem to propose quality dimensions (e.g., if they talk about practices)

%Query: \footnote{\url{https://www.scopus.com/results/results.uri?sort=plf-f&src=s&nlo=&nlr=&nls=&sid=b04451f7b887660ce99d73bfdbdc4fc8&sot=a&sdt=a&cluster=scosubjabbr%2c%22COMP%22%2ct%2c%22ENGI%22%2ct%2bscoexactsrctitle%2c%22Software+Quality+Journal%22%2ct%2c%22Proceedings+International+Conference+On+Software+Engineering%22%2ct%2c%22IEEE+Transactions+on+Software+Engineering%22%2ct%2c%22Empirical+Software+Engineering%22%2ct%2c%22Journal+of+Systems+and+Software%22%2ct%2c%22Software+%26+Systems+Modeling%22%2ct%2c%22Information+and+Software+Technology%22%2ct%2c%22IEEE+Software%22%2ct&sl=30&s=TITLE+%28+software+AND+quality+%29&cl=t&offset=201&origin=resultslist&ss=plf-f&ws=r-f&ps=r-f&cs=r-f&cc=10&txGid=cff6749bf8a084912200dbef379950d4} 


%Also, this book: https://books.google.es/books?hl=en&lr=&id=XTvpAQAAQBAJ&oi=fnd&pg=PR3&dq=software+quality&ots=fohz_-KW0d&sig=5TGlvR3sgAIkAHzs5Iup8Qijpuo#v=onepage&q=software%20quality&f=false looks nice!